\documentclass{article}
\usepackage{amsmath}
\usepackage[utf8]{inputenc}
\usepackage{graphicx} % Comandos para manejar imágenes
\graphicspath{ {./images/} } % Carpeta de imágenes
\usepackage[table,xcdraw]{xcolor}
\setlength{\parskip}{2mm} % Espaciado

\usepackage[utf8]{inputenc}
\usepackage{geometry}
    \geometry{left=3cm,right=2cm,top=2cm,bottom=2cm}
%
\usepackage[spanish]{babel}
%
\usepackage[fixlanguage]{babelbib}
    \bibliographystyle{babunsrt}
%

\usepackage{floatrow}
\floatsetup[table]{style=plaintop}

\usepackage{url}

\usepackage[top=2cm, bottom=2.5cm, right=3 cm, left=3 cm]{geometry} % margenes

\usepackage{parskip} % Sangria

\title{Informe XX \textbf{Lab}oratorio de Máquinas: nombre de la experiencia}
\author{Nombre y Apellidos$^{1}$\\
\small{$^{1}$Escuela de Ingeniería Mecánica}\\
\small{Pontificia Universidad Católica de Valparaíso}\\
\small{cristobal.galleguillos@pucv.cl}
}
\date{\small{\today}}

\begin{document}

\maketitle

\section{Introducción}

Breve descripción de no más de tres párrafos sobre lo que se realizó, observó o investigó. 

Debe indicar siempre el objetivo general (la pregunta que quiere responder) y al menos dos y máximo cinco objetivos específicos (como responderá esa pregunta)


\section{Revisión de la literatura}

Dependiendo del tipo de informe debe considerar incluir acá las ecuaciones, fundamentación teórica de su trabajo y otros ejemplos relacionados con su trabajo (informe técnico).

Si es un informe ejecutivo, puede hacer referencias a otros estudios, recortes de prensa o artículos de opinión.

La extensión de este apartado no debiese ser mas de una página, considerando que todo está debidamente referenciado.

\section{Desarrollo}

Con un desarrollo secuencial y lógico de cada uno de los objetivos específicos, busca una respuesta a la pregunta inicial del informe, usa gráficos, tablas, desarrolla ecuaciones o usa ejemplos, siempre indicando la referencia.

Dependiendo de los gráficos obtenidos la extensión podría ser máximo unas cinco páginas.

Los datos son anexos y se entregan tal como los entregamos a ustedes, en un archivo de preferencia plano, y se sube al repositorio.

\section{Conclusiones}

¿Se respondio la pregunta principal?¿Qué desviaciones?¿Que propone?

Se evaluará mucho su juicio critico.

\nocite{*}
    \bibliography{src/ref}

\end{document}
